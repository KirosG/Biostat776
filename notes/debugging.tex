\documentclass[aspectratio=169]{beamer}

\mode<presentation>
{
  \usetheme{Warsaw}
  % or ...

  \setbeamercovered{transparent}
  % or whatever (possibly just delete it)
}


\usepackage[english]{babel}
\usepackage[latin1]{inputenc}
\usepackage{graphicx}

\usepackage{amsmath,amsfonts,amssymb}

\newcommand{\pkg}{\textbf}
\newcommand{\code}{\texttt}
\newcommand{\bsym}{\boldsymbol}
\newcommand{\eps}{\varepsilon}
\newcommand{\bx}{\mathbf{x}}
\newcommand{\bX}{\mathbf{X}}
\newcommand{\bZ}{\mathbf{Z}}
\newcommand{\bz}{\mathbf{z}}
\newcommand{\newp}{\vspace{1em}}
\newcommand{\Var}{\text{Var}}
\newcommand{\PMTen}{PM$_{10}$}
\newcommand{\Cov}{\text{Cov}}
\newcommand{\bi}{\begin{itemize}}
\newcommand{\ei}{\end{itemize}}


\title[Debugging]{Debugging}


\date{Biostatistics 140.776}

\setbeamertemplate{footline}[page number]

\begin{document}

\begin{frame}
  \titlepage
\end{frame}


\begin{frame}{Something's Wrong!}
Indications that something's not right
\begin{itemize}
\item \code{message}: A generic notification/diagnostic message
  produced by the \code{message} function; execution of the function
  continues
\item \code{warning}: An indication that something is wrong but not
  necessarily fatal; execution of the function continues; generated by
  the \code{warning} function
\item \code{error}: An indication that a fatal problem has occurred;
  execution stops; produced by the \code{stop} function
\item \code{condition}: A generic concept for indicating that
  something unexpected can occur; programmers can create their own
  conditions
\end{itemize}
\end{frame}

\begin{frame}[fragile]{Something's Wrong!}
Warning
\begin{verbatim}
> log(-1)
[1] NaN
Warning message:
In log(-1) : NaNs produced
\end{verbatim}
\end{frame}

\begin{frame}[fragile]{Something's Wrong}
\begin{verbatim}
printmessage <- function(x) {
        if(x > 0)
                print("x is greater than zero")
        else 
                print("x is less than or equal to zero")
        invisible(x)
}
\end{verbatim}
\end{frame}

\begin{frame}[fragile]{Something's Wrong}
\begin{verbatim}
printmessage <- function(x) {
        if(x > 0)
                print("x is greater than zero")
        else 
                print("x is less than or equal to zero")
        invisible(x)
}
> printmessage(1)
[1] "x is greater than zero"
> printmessage(NA)
Error in if (x > 0) { : missing value where TRUE/FALSE needed
\end{verbatim}
\end{frame}

\begin{frame}[fragile]{Something's Wrong!}
\begin{verbatim}
printmessage2 <- function(x) {
        if(is.na(x))
                print("x is a missing value!")
        else if(x > 0)
                print("x is greater than zero")
        else 
                print("x is less than or equal to zero")
        invisible(x)
}
\end{verbatim}
\end{frame}

\begin{frame}[fragile]{Something's Wrong!}
\begin{verbatim}
printmessage2 <- function(x) {
        if(is.na(x))
                print("x is a missing value!")
        else if(x > 0)
                print("x is greater than zero")
        else 
                print("x is less than or equal to zero")
        invisible(x)
}
> x <- log(-1)
Warning message:
In log(-1) : NaNs produced
> printmessage2(x)
[1] "x is a missing value!"
\end{verbatim}
\end{frame}

\begin{frame}{Something's Wrong!}
How do you know that something is wrong with your function?
\begin{itemize}
\item What was your input? How did you call the function?
\item What were you expecting? Output, messages, other results?
\item What did you get?
\item How does what you get differ from what you were expecting?
\item Were your expectations correct in the first place?
\item Can you reproduce the problem (exactly)?
\end{itemize}
\end{frame}

\begin{frame}{Debugging Tools in R}
The primary tools for debugging functions in R are
\begin{itemize}
\item \code{traceback}: prints out the function call stack after an
  error occurs; does nothing if there's no error
\item \code{debug}: flags a function for ``debug'' mode which allows
  you to step through execution of a function one line at a time
\item \code{browser}: suspends the execution of a function wherever it
  is called and puts the function in debug mode 
\item \code{trace}: allows you to insert debugging code into a
  function a specific places
\item \code{recover}: allows you to modify the error behavior so that
  you can browse the function call stack
\end{itemize}
These are interactive tools specifically designed to allow you to pick
through a function. There's also the more blunt technique of inserting
\code{print}/\code{cat} statements in the function.
\end{frame}




\begin{frame}[fragile]{traceback}
\begin{verbatim}
> mean(x)
Error in mean(x) : object 'x' not found
> traceback()
1: mean(x)
> 
\end{verbatim}
\end{frame}

\begin{frame}[fragile]{traceback}
\begin{verbatim}
> lm(y ~ x)
Error in eval(expr, envir, enclos) : object 'y' not found
> traceback()
7: eval(expr, envir, enclos)
6: eval(predvars, data, env)
5: model.frame.default(formula = y ~ x, drop.unused.levels = TRUE)
4: model.frame(formula = y ~ x, drop.unused.levels = TRUE)
3: eval(expr, envir, enclos)
2: eval(mf, parent.frame())
1: lm(y ~ x)
\end{verbatim}
\end{frame}

\begin{frame}[fragile]{debug}
\begin{verbatim}
> debug(lm)
> lm(y ~ x)
debugging in: lm(y ~ x)
debug: {
    ret.x <- x
    ret.y <- y
    cl <- match.call()
    ...
    if (!qr) 
        z$qr <- NULL
    z
}
Browse[2]> 
\end{verbatim}
\end{frame}

\begin{frame}[fragile]{debug}
\begin{verbatim}
Browse[2]> n
debug: ret.x <- x
Browse[2]> n
debug: ret.y <- y
Browse[2]> n
debug: cl <- match.call()
Browse[2]> n
debug: mf <- match.call(expand.dots = FALSE)
Browse[2]> n
debug: m <- match(c("formula", "data", "subset", "weights", "na.action", 
    "offset"), names(mf), 0L)
\end{verbatim}
\end{frame}

\begin{frame}[fragile]{recover}
\begin{verbatim}
> options(error = recover)
> read.csv("nosuchfile")
Error in file(file, "rt") : cannot open the connection
In addition: Warning message:
In file(file, "rt") :
  cannot open file 'nosuchfile': No such file or directory

Enter a frame number, or 0 to exit   

1: read.csv("nosuchfile")
2: read.table(file = file, header = header, sep = sep, quote = quote, dec = de
3: file(file, "rt")

Selection: 
\end{verbatim}
\end{frame}

\begin{frame}{Debugging}
Summary
\begin{itemize}
  \item There are three main indications of a problem/condition:
    message, warning, error; only an error is fatal
  \item When analyzing a function with a problem, make sure you can
    reproduce the problem, clearly state your expectations and how the
    output differs from your expectation
  \item Interactive debugging tools \code{traceback}, \code{debug},
    \code{browser}, \code{trace}, and \code{recover} can be used to
    find problematic code in functions
  \item Debugging tools are not a substitute for thinking!
\end{itemize}
\end{frame}


\end{document}


