\documentclass[aspectratio=169]{beamer}

\mode<presentation>
{
  \usetheme{Warsaw}
  % or ...

  \setbeamercovered{transparent}
  % or whatever (possibly just delete it)
}


\usepackage[english]{babel}
\usepackage[latin1]{inputenc}
\usepackage{graphicx}

\usepackage{amsmath,amsfonts,amssymb}

\newcommand{\pkg}{\textbf}
\newcommand{\code}{\texttt}
\newcommand{\bsym}{\boldsymbol}
\newcommand{\eps}{\varepsilon}
\newcommand{\bx}{\mathbf{x}}
\newcommand{\bX}{\mathbf{X}}
\newcommand{\bZ}{\mathbf{Z}}
\newcommand{\bz}{\mathbf{z}}
\newcommand{\newp}{\vspace{1em}}
\newcommand{\Var}{\text{Var}}
\newcommand{\PMTen}{PM$_{10}$}
\newcommand{\Cov}{\text{Cov}}
\newcommand{\bi}{\begin{itemize}}
\newcommand{\ei}{\end{itemize}}


\title[The R Language]{Introduction to the R Language}

\subtitle{Data Types and Basic Operations}

\date{Biostatistics 140.776}

\setbeamertemplate{footline}[page number]


\begin{document}

\begin{frame}
  \titlepage
\end{frame}


\begin{frame}{Objects}
R has five basic or ``atomic'' classes of objects:
\begin{itemize}
\item
character
\item
numeric (real numbers)
\item
integer
\item
complex
\item
logical  (True/False)
\end{itemize}
The most basic object is a vector
\begin{itemize}
\item
A vector can only contain objects of the same class
\item
BUT: The one exception is a \textit{list}, which is represented as a
vector but can contain objects of different classes (indeed, that's
usually why we use them)
\end{itemize}
Empty vectors can be created with the \code{vector()} function.
\end{frame}

\begin{frame}{Numbers}
\begin{itemize}
\item
Numbers in R a generally treated as numeric objects (i.e. double
precision real numbers)
\item
If you explicitly want an integer, you need to specify the \code{L}
suffix
\item
Ex: Entering \code{1} gives you a numeric object; entering \code{1L}
explicitly gives you an integer. 
\item
There is also a special number \code{Inf} which represents infinity;
e.g.  \code{1 / 0}; \code{Inf} can be used in ordinary calculations;
e.g. \code{1 / Inf} is 0
\item
The value \code{NaN} represents an undefined value (``not a number'');
e.g. 0 / 0; \code{NaN} can also be thought of as a missing value (more
on that later)
\end{itemize}
\end{frame}

\begin{frame}{Attributes}
R objects can have attributes
\begin{itemize}
\item
names, dimnames
\item
dimensions (e.g. matrices, arrays)
\item
class
\item
length
\item
other user-defined attributes/metadata
\end{itemize}
Attributes of an object can be accessed using the \code{attributes()}
function.
\end{frame}

\begin{frame}[fragile]{Entering Input}
At the R prompt we type \textit{expressions}.  The \code{<-} symbol is
the assignment operator.
\begin{verbatim}
> x <- 1
> print(x)
[1] 1
> x
[1] 1
> msg <- "hello"
\end{verbatim}
The grammar of the language determines whether an expression is
complete or not.
\begin{verbatim}
> x <-  ## Incomplete expression
\end{verbatim}
The \code{\#} character indicates a \textit{comment}.  Anything to the
right of the \code{\#} (including the \code{\#} itself) is ignored.
\end{frame}

\begin{frame}[fragile]{Evaluation}
When a complete expression is entered at the prompt, it is
\textit{evaluated} and the result of the evaluated expression is
returned.  The result may be \textit{auto-printed}.
\begin{verbatim}
> x <- 5  ## nothing printed
> x       ## auto-printing occurs
[1] 5
> print(x)  ## explicit printing
[1] 5
\end{verbatim}
The \code{[1]} indicates that \code{x} is a vector and 5 is the first
element.
\end{frame}


\begin{frame}[fragile]{Printing}
\begin{verbatim}
> x <- 1:20
> x
 [1]  1  2  3  4  5  6  7  8  9 10 11 12 13 14 15
[16] 16 17 18 19 20
\end{verbatim}
The \code{:} operator is used to create integer sequences.
\end{frame}

\begin{frame}[fragile]{Creating Vectors}
The \code{c()} function can be used to create vectors of objects.
\begin{verbatim}
> x <- c(0.5, 0.6)       ## numeric
> x <- c(TRUE, FALSE)    ## logical
> x <- c(T, F)           ## logical
> x <- c("a", "b", "c")  ## character
> x <- 9:29              ## integer
> x <- c(1+0i, 2+4i)     ## complex
\end{verbatim}
Using the \code{vector()} function
\begin{verbatim}
> x <- vector("numeric", length = 10)
> x
 [1] 0 0 0 0 0 0 0 0 0 0
\end{verbatim}
\end{frame}

\begin{frame}[fragile]{Mixing Objects}
What about the following?
\begin{verbatim}
> y <- c(1.7, "a")   ## character
> y <- c(TRUE, 2)    ## numeric
> y <- c("a", TRUE)  ## character
\end{verbatim}
When different objects are mixed in a vector, \textit{coercion} occurs
so that every element in the vector is of the same class.
\end{frame}

\begin{frame}[fragile]{Explicit Coercion}
Objects can be explicitly coerced from one class to another using the
\code{as.*} functions, if available.
\begin{verbatim}
> x <- 0:6
> class(x)
[1] "integer"
> as.numeric(x)
[1] 0 1 2 3 4 5 6
> as.logical(x)
[1] FALSE  TRUE  TRUE  TRUE  TRUE  TRUE  TRUE
> as.character(x)
[1] "0" "1" "2" "3" "4" "5" "6"
> as.complex(x)
[1] 0+0i 1+0i 2+0i 3+0i 4+0i 5+0i 6+0i
\end{verbatim}
\end{frame}

\begin{frame}[fragile]{Explicit Coercion}
Nonsensical coercion results in \code{NA}s.
\begin{verbatim}
> x <- c("a", "b", "c")
> as.numeric(x)
[1] NA NA NA
Warning message:
NAs introduced by coercion 
> as.logical(x)
[1] NA NA NA
\end{verbatim}
\end{frame}


\begin{frame}[fragile]{Matrices}
Matrices are vectors with a \textit{dimension} attribute.  The
dimension attribute is itself an integer vector of length 2 (nrow,
ncol)
\begin{verbatim}
> m <- matrix(nrow = 2, ncol = 3)
> m
     [,1] [,2] [,3]
[1,]   NA   NA   NA
[2,]   NA   NA   NA
> dim(m)
[1] 2 3
> attributes(m)
$dim
[1] 2 3
\end{verbatim}
\end{frame}

\begin{frame}[fragile]{Matrices (cont'd)}
Matrices are constructed \textit{column-wise}, so entries can be
thought of starting in the ``upper left'' corner and running down the
columns.
\begin{verbatim}
> m <- matrix(1:6, nrow = 2, ncol = 3)
> m
     [,1] [,2] [,3]
[1,]    1    3    5
[2,]    2    4    6
\end{verbatim}
\end{frame}

\begin{frame}[fragile]{Matrices (cont'd)}
Matrices can also be created directly from vectors by adding a
dimension attribute.
\begin{verbatim}
> m <- 1:10
> m
 [1]  1  2  3  4  5  6  7  8  9 10
> dim(m) <- c(2, 5)
> m
     [,1] [,2] [,3] [,4] [,5]
[1,]    1    3    5    7    9
[2,]    2    4    6    8   10
\end{verbatim}
\end{frame}

\begin{frame}[fragile]{cbind-ing and rbind-ing}
Matrices can be created by \textit{column-binding} or
\textit{row-binding} with \code{cbind()} and \code{rbind()}.
\begin{verbatim}
> x <- 1:3
> y <- 10:12
> cbind(x, y)
     x  y
[1,] 1 10
[2,] 2 11
[3,] 3 12
> rbind(x, y)
  [,1] [,2] [,3]
x    1    2    3
y   10   11   12
\end{verbatim}
\end{frame}

\begin{frame}[fragile]{Lists}
Lists are a special type of vector that can contain elements of
different classes.  Lists are a very important data type in R and you
should get to know them well.
\begin{verbatim}
> x <- list(1, "a", TRUE, 1 + 4i)
> x
[[1]]
[1] 1

[[2]]
[1] "a"

[[3]]
[1] TRUE

[[4]]
[1] 1+4i
\end{verbatim}
\end{frame}

\begin{frame}{Factors}
Factors are used to represent categorical data.  Factors can be
unordered or ordered.  One can think of a factor as an integer vector
where each integer has a \textit{label}.
\begin{itemize}
\item
Factors are treated specially by modelling functions like \code{lm()}
and \code{glm()}
\item
Using factors with labels is \textit{better} than using integers
because factors are self-describing; having a variable that has values
``Male'' and ``Female'' is better than a variable that has values 1
and 2.
\end{itemize}
\end{frame}

\begin{frame}[fragile]{Factors}
\begin{verbatim}
> x <- factor(c("yes", "yes", "no", "yes", "no"))
> x
[1] yes yes no  yes no 
Levels: no yes
> table(x)
x
 no yes 
  2   3 
> unclass(x)
[1] 2 2 1 2 1
attr(,"levels")
[1] "no"  "yes"
\end{verbatim}
\end{frame}


\begin{frame}[fragile]{Factors}
The order of the levels can be set using the \code{levels} argument to
\code{factor()}.  This can be important in linear modelling because
the first level is used as the baseline level.
\begin{verbatim}
> x <- factor(c("yes", "yes", "no", "yes", "no"), 
              levels = c("yes", "no"))
> x
[1] yes yes no  yes no 
Levels: yes no
\end{verbatim}
\end{frame}

\begin{frame}[fragile]{Missing Values}
Missing values are denoted by \code{NA} or \code{NaN} for undefined
mathematical operations.
\begin{itemize}
\item
\code{is.na()} is used to test objects if they are \code{NA}
\item
\code{is.nan()} is used to test for \code{NaN}
\item
\code{NA} values have a class also, so there are integer \code{NA},
character \code{NA}, etc.  
\item
A \code{NaN} value is also \code{NA} but the converse is not true
\end{itemize}
\end{frame}

\begin{frame}[fragile]{Missing Values}
\begin{verbatim}
> x <- c(1, 2, NA, 10, 3)
> is.na(x)
[1] FALSE FALSE  TRUE FALSE FALSE
> is.nan(x)
[1] FALSE FALSE FALSE FALSE FALSE
> x <- c(1, 2, NaN, NA, 4)
> is.na(x)
[1] FALSE FALSE  TRUE  TRUE FALSE
> is.nan(x)
[1] FALSE FALSE  TRUE FALSE FALSE
\end{verbatim}
\end{frame}

\begin{frame}{Data Frames}
Data frames are used to store tabular data
\begin{itemize}
\item
They are represented as a special type of list where every element of
the list has to have the same length
\item
Each element of the list can be thought of as a column and the length
of each element of the list is the number of rows
\item
Unlike matrices, data frames can store different classes of objects in
each column (just like lists); matrices must have every element be the
same class
\item
Data frames also have a special attribute called \code{row.names}
\item
Data frames are usually created by calling \code{read.table()} or
\code{read.csv()}
\item
Can be converted to a matrix by calling \code{data.matrix()}
\end{itemize}
\end{frame}

\begin{frame}[fragile]{Data Frames}
\begin{verbatim}
> x <- data.frame(foo = 1:4, bar = c(T, T, F, F))
> x
  foo   bar
1   1  TRUE
2   2  TRUE
3   3 FALSE
4   4 FALSE
> nrow(x)
[1] 4
> ncol(x)
[1] 2
\end{verbatim}
\end{frame}

\begin{frame}[fragile]{Names}
R objects can also have names, which is very useful for writing
readable code and self-describing objects.
\begin{verbatim}
> x <- 1:3
> names(x)
NULL
> names(x) <- c("foo", "bar", "norf")
> x
 foo  bar norf 
   1    2    3 
> names(x)
[1] "foo"  "bar"  "norf"
\end{verbatim}
\end{frame}

\begin{frame}[fragile]{Names}
Lists can also have names.
\begin{verbatim}
> x <- list(a = 1, b = 2, c = 3)
> x
$a
[1] 1

$b
[1] 2

$c
[1] 3
\end{verbatim}
\end{frame}

\begin{frame}[fragile]{Names}
And matrices.
\begin{verbatim}
> m <- matrix(1:4, nrow = 2, ncol = 2)
> dimnames(m) <- list(c("a", "b"), c("c", "d"))
> m
  c d
a 1 3
b 2 4
\end{verbatim}
\end{frame}

\begin{frame}[fragile]{Summary}
Data Types
\begin{itemize}
\item atomic classes: numeric, logical, character, integer, complex
\item vectors, lists
\item factors
\item missing values
\item data frames
\item names
\end{itemize}
\end{frame}




\begin{frame}[fragile]{Subsetting}
There are a number of operators that can be used to extract subsets of
R objects.
\begin{itemize}
\item
\verb+[+ always returns an object of the same class as the original;
can be used to select more than one element (there is one exception)
\item
\verb+[[+ is used to extract elements of a list or a data frame; it
can only be used to extract a single element and the class of the
returned object will not necessarily be a list or data frame
\item
\verb+$+ is used to extract elements of a list or data frame by name;
semantics are similar to hat of \verb+[[+.
\end{itemize}
\end{frame}

\begin{frame}[fragile]{Subsetting}
\begin{verbatim}
> x <- c("a", "b", "c", "c", "d", "a")
> x[1]
[1] "a"
> x[2]
[1] "b"
> x[1:4]
[1] "a" "b" "c" "c"
> x[x > "a"]
[1] "b" "c" "c" "d"
> u <- x > "a"
> u
[1] FALSE  TRUE  TRUE  TRUE  TRUE FALSE
> x[u]
[1] "b" "c" "c" "d"
\end{verbatim}
\end{frame}

\begin{frame}[fragile]{Subsetting a Matrix}
Matrices can be subsetted in the usual way with $(i, j)$ type indices.
\begin{verbatim}
> x <- matrix(1:6, 2, 3)
> x[1, 2]
[1] 3
> x[2, 1]
[1] 2
\end{verbatim}
Indices can also be missing.
\begin{verbatim}
> x[1, ]
[1] 1 3 5
> x[, 2]
[1] 3 4
\end{verbatim}
\end{frame}

\begin{frame}[fragile]{Subsetting a Matrix}
By default, when a single element of a matrix is retrieved, it is
returned as a vector of length 1 rather than a $1\times 1$ matrix.
This behavior can be turned off by setting \code{drop = FALSE}.
\begin{verbatim}
> x <- matrix(1:6, 2, 3)
> x[1, 2]
[1] 3

> x[1, 2, drop = FALSE]
     [,1]
[1,]    3
\end{verbatim}
\end{frame}

\begin{frame}[fragile]{Subsetting a Matrix}
Similarly, subsetting a single column or a single row will give you a
vector, not a matrix (by default).
\begin{verbatim}
> x <- matrix(1:6, 2, 3)
> x[1, ]
[1] 1 3 5
> x[1, , drop = FALSE]
     [,1] [,2] [,3]
[1,]    1    3    5
\end{verbatim}
\end{frame}

\begin{frame}[fragile]{Subsetting Lists}
\begin{verbatim}
> x <- list(foo = 1:4, bar = 0.6)
> x[1]
$foo
[1] 1 2 3 4

> x[[1]]
[1] 1 2 3 4

> x$bar
[1] 0.6
> x[["bar"]]
[1] 0.6
> x["bar"]
$bar
[1] 0.6
\end{verbatim}
\end{frame}

\begin{frame}[fragile]{Subsetting Lists}
Extracting multiple elements of a list.
\begin{verbatim}
> x <- list(foo = 1:4, bar = 0.6, baz = "hello")
> x[c(1, 3)]
$foo
[1] 1 2 3 4

$baz
[1] "hello"
\end{verbatim}
\end{frame}

\begin{frame}[fragile]{Subsetting Lists}
The \verb+[[+ operator can be used with \textit{computed} indices;
\verb+$+ can only be used with literal names.
\begin{verbatim}
> x <- list(foo = 1:4, bar = 0.6, baz = "hello")
> name <- "foo"
> x[[name]]  ## computed index for `foo'
[1] 1 2 3 4
> x$name     ## element `name' doesn't exist!
NULL
> x$foo
[1] 1 2 3 4  ## element `foo' does exist
\end{verbatim}
\end{frame}

\begin{frame}[fragile]{Subsetting Nested Elements of a List}
The \verb+[[+ can take an integer sequence.
\begin{verbatim}
> x <- list(a = list(10, 12, 14), b = c(3.14, 2.81))
> x[[c(1, 3)]]
[1] 14
> x[[1]][[3]]
[1] 14

> x[[c(2, 1)]]
[1] 3.14
\end{verbatim}
\end{frame}

\begin{frame}[fragile]{Partial Matching}
Partial matching of names is allowed with \verb+[[+ and \verb+$+.
\begin{verbatim}
> x <- list(aardvark = 1:5)
> x$a
[1] 1 2 3 4 5
> x[["a"]]
NULL
> x[["a", exact = FALSE]]
[1] 1 2 3 4 5
\end{verbatim}
\end{frame}

\begin{frame}[fragile]{Removing NA Values}
A common task is to remove missing values (\code{NA}s).
\begin{verbatim}
> x <- c(1, 2, NA, 4, NA, 5)
> bad <- is.na(x)
> x[!bad]
[1] 1 2 4 5
\end{verbatim}
\end{frame}

\begin{frame}[fragile]{Removing NA Values}
What if there are multiple things and you want to take the subset with
no missing values?
\begin{verbatim}
> x <- c(1, 2, NA, 4, NA, 5)
> y <- c("a", "b", NA, "d", NA, "f")
> good <- complete.cases(x, y)
> good
[1]  TRUE  TRUE FALSE  TRUE FALSE  TRUE
> x[good]
[1] 1 2 4 5
> y[good]
[1] "a" "b" "d" "f"
\end{verbatim}
\end{frame}

\begin{frame}[fragile]{Removing NA Values}
\begin{verbatim}
> airquality[1:6, ]
  Ozone Solar.R Wind Temp Month Day
1    41     190  7.4   67     5   1
2    36     118  8.0   72     5   2
3    12     149 12.6   74     5   3
4    18     313 11.5   62     5   4
5    NA      NA 14.3   56     5   5
6    28      NA 14.9   66     5   6
> good <- complete.cases(airquality)
> airquality[good, ][1:6, ]
  Ozone Solar.R Wind Temp Month Day
1    41     190  7.4   67     5   1
2    36     118  8.0   72     5   2
3    12     149 12.6   74     5   3
4    18     313 11.5   62     5   4
7    23     299  8.6   65     5   7
8    19      99 13.8   59     5   8
\end{verbatim}
\end{frame}


\begin{frame}[fragile]{Vectorized Operations}
Many operations in R are \textit{vectorized} making code more
efficient, concise, and easier to read.
\begin{verbatim}
> x <- 1:4; y <- 6:9
> x + y
[1]  7  9 11 13
> x > 2
[1] FALSE FALSE  TRUE  TRUE
> x >= 2
[1] FALSE  TRUE  TRUE  TRUE
> y == 8
[1] FALSE FALSE  TRUE FALSE
> x * y
[1]  6 14 24 36
> x / y
[1] 0.1666667 0.2857143 0.3750000 0.4444444
\end{verbatim}
\end{frame}

\begin{frame}[fragile]{Vectorized Matrix Operations}
\begin{verbatim}
> x <- matrix(1:4, 2, 2); y <- matrix(rep(10, 4), 2, 2)
> x * y       ## element-wise multiplication
     [,1] [,2]
[1,]   10   30
[2,]   20   40
> x / y
     [,1] [,2]
[1,]  0.1  0.3
[2,]  0.2  0.4
> x %*% y     ## true matrix multiplication
     [,1] [,2]
[1,]   40   40
[2,]   60   60
\end{verbatim}
\end{frame}


\end{document}
